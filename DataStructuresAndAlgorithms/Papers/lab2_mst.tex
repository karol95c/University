\documentclass[a4paper,10pt]{article}
\usepackage{tabto}
\usepackage{listings}
\usepackage{tikz}
\usepackage[T1]{fontenc}
\usepackage[polish]{babel}
\usepackage[utf8]{inputenc}
\usepackage{lmodern}
\selectlanguage{polish}
\usepackage{amsmath}
\usepackage{amsthm}
\author{Karol Cidyło}
\title {Zadanie 6 z listy 1.}
\usetikzlibrary{matrix}
\begin{document}
Karol Cidyło \newline \newline \newline
Zadanie 6 z listy 2. \newline \newline
Lemat cycle property. \newline
Weźmy cykl C w grafie. Jeśli waga krawędzi e z C jest największą wagą spośród krawędzi cyklu (z pozostałych wierzchołków zbioru C) wtedy ta krawędź nie należy do MST. \newline \newline
Dowód: \newline
Załóżmy, że e należy do MST $T_1$, usunięcie e powoduję cięcie $T_1$ na dwa poddrzewa z dwoma końcami e w różnych poddrzewach.\newline Skoro mieliśmy cykl to możemy znealeźć krawędź f z C, która połączy nam nasze początkowe drzewo. Jeśli e jest maksymalne to w(f) < w(e), więc zastępując krawędź e przez krawędź f dostajemy MST $T_2$ o mniejszejs wadze niż MST $T_1$.  \newline \newline
Idea algorytmu:
Weźmy krawędź e($v_1$, $v_2$). Zapiszmy wagę i usuńmy z grafu. \newline
Używając zmienionego DFS przejdziemy krawędzie pod warunkiem, że mają one mniejszą wagę od zapamiętanej wagi krawędzi e($v_1$, $v_2$). Jeśli zaczniemy z $v_1$ i dotrzemy do $v_2$ to znaczy, że e należało do cyklu i w(e) było największą wagą z cyklu(przchodziliśmy tylko krawędzie mniejsze od w(e), dodając e stworzymy cyjl). Korzystając z cycle property wiemy, że e nie należy do MST. Jeśl przejdziemy cały graf(poza krawędzią e) i i nie dotrzemy do $v_2$ to znaczy, że e należy do MST. \newline \newline
Pseudokod: \newline
\begin{lstlisting}
maks_waga = waga(e)

dfs(v_start, v_koniec):
    dla kazdego sasiada u wierzcholka v_start:
        jesli (u == v_koniec) zwroc falsz
        jesli odwiedzony[u] == 0 i waga(e(v,u))< maks_waga
        dfs(u, v_koniec)
    zwroc prawda
    

\end{lstlisting}
Algorytm działa w czasie O(n+m) tak jak tradycyjny dfs. Przchodzimy każdą krawędź  oraz wierzchołek raz.  \newline 
Może nawet zakończyć się szybciej, jeśli dojdziemy do wierzchołka $v_2$. \newline\newline
Poniżej zamieszczam dokładniejszą implementację wykorzystującą listę sąsiedztwa wierzchołków w grafie 
oraz iteracyjnego dfs-a, według mnie jest to najlepsze rozwiązanie, ponieważ pozwala łatwo zakończyć działanie 
po dotarciu do żądanego wierzchołka.\newline
Przepraszam za niespójność w języku wyrażeń. Niżej używam angielskiego, ponieważ kod naturalniej pisze mi się używając angielskich nazw oraz zwrotów.
\newpage
\begin{lstlisting}
dfs(e):
    max_weight = weight(e)
    stack.push(e.start)
	
    while (!stack.empty())
    {
        s = stack.top();
        stack.pop();
		
        if (!visited[s])
        {
            visited[s] = true;
        }
		
        for (auto i = adj[s].begin(); i != adj[s].end(); ++i) 
            if (*i == e.end) return false;
            if (!visited[*i] && weight(s, *i) < max_weight)
                stack.push(*i);
        }
    }

    return true;
}
	

\end{lstlisting}

\end{document}