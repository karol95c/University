\documentclass[a4paper,10pt]{article}
\usepackage{tabto}
\usepackage{listings}
\usepackage{tikz}
\usepackage{enumitem}
\usepackage[T1]{fontenc}
\usepackage[polish]{babel}
\usepackage[utf8]{inputenc}
\usepackage{lmodern}
\selectlanguage{polish}
\usepackage{amsmath}
\usepackage{amsthm}
\usepackage{MnSymbol}
\usepackage{graphicx}
\author{Karol Cidyło}
\title {Zadanie 6 z listy4.}
\usetikzlibrary{matrix}
\begin{document}
Karol Cidyło \newline \newline \newline
Zadanie 6 z listy4. \newline \newline
Mamy dwa podciągi x oraz y. Zadaniem jest znalezienie najdłuższego wspólnego podciągu nie zawierającego słowa 'egzamin'.
Rozwiązanie wykorzystuje programowanie dynamiczne. Wykorzystuje również tablice dwuwymiarowe gdzie T[i][j] to długość najdłuższego wspólnego podciągu
złożonego z prefiksów x[1...i], y[1...j]. 
Aby otrzymać rozwiązanie dla podciągu nie zawierającego słowa `egzamin` dla danych x i y oraz i,j oznaczających indeksy znaków w danym ciągu 
będziemy obliczać tablice dla rozwiązań pośrednich.  To znaczy: \newline
${T_1}$ jest tablicą gdzie trzymane są wyniki najdłuższego wspólnego podciągu nie zawierającego podsłowa `egzamin' i nie kończy się ten podciąg na `e` \newline
${T_2}$ jest tablicą gdzie trzymane są wyniki najdłuższego wspólnego podciągu nie zawierającego podsłowa `egzamin' i nie kończy się ten podciąg na `eg`\newline
I tak dalej aż do:
${T_7}$ czyli wyniki najdłuższego wspólnego podciągu nie zawierającego podsłowa `egzamin' \newline
Przyjmjmy kiedy mówimy o indeksach i-tych mamy na myśli i-ty znak ciągu x, a kiedy o j-tych to j-ty znak ciągu y. \newline

Funkcja obliczT1(i, j) będzie obliczać pole ${T_1}$[i][j].
\begin{lstlisting}[
    basicstyle=\tiny, %or \small or \footnotesize etc.
    tabsize=4
]
obliczT1(i,j):
	jesli i == 0 || j == 0:
		T1[i][j] = 0;

	jesli x[i] != y[j]:					(1)
		T1[i][j] = max(T1[i-1][j], T1[i][j-1])

	jesli x[i] == y[j] && x[i] == `e`:  			(2)
		T1[i][j] = T1[i-1][j-1]

	jesli x[i] == y[j] && x[i] == `n`:  			(3)
		T1[i][j] = max(T6[i-1][j-1] + 1, T1[i-1][j-1])

	jesli x[i] == y[j] && x[i] != `e` &&  x[i] != `n`: 	(4)
		T1[i][j] = T7[i-1][j-1] + 1


\end{lstlisting}
\textbf{(1)} - znaki na miejsach i oraz j są różne, więc naturalnie bierzemy pod uwagę maksymalne wyniki dla innych indeksów obu ciągów. \newline
\textbf{(2)} - $T_1$ zawiera takie wyniki, że najdłuższy wspólny podciąg nie kończy się na `e`, więc nie możemy dodać wystąpienia litery w obu ciągach do wyniku. \newline \newline
\textbf{(3)} - znaki na miejsach i oraz j są równe `n` czyli możemy wybrać maksymalny wynik z \textbf{I.} najdłuższych ciągów takich, które nie kończą się na `egzami', ponieważ dodając n, mielibyśmy całe słowo `egzamin`(co nie może być brane pod uwagę przy obliczaniu wyniku. Do tego dodajemy 1, ponieważ możemy dodać n i nie będziemy mieć słowa `egzamin` oraz litery `e` na końcu) oraz \textbf{II.} ciągiem nie kończącym się na `e` ale dla obu indeksów mniejszych o 1. \newline  \newline
\textbf{(4)} - jeśli wiemy, że litera która jest wspólna dla obu ciągów nie jest `e` oraz `n` bierzemy wynik z  T7[i-1][j-1] i dodajemy do niego 1, czyli zwiększamy wynik, ponieważ nie psuje nam to założeń zadania oraz T1. \newpage
Natomiast funkcja oblicz T2(i,j) wygląda bardzo podobnie:
\begin{lstlisting}[
    basicstyle=\tiny, %or \small or \footnotesize etc.
    tabsize=4
]
obliczT2(i,j):
	jesli i == 0 || j == 0:
		T2[i][j] = 0;

	jesli x[i] != y[j]:
		T2[i][j] = max(T2[i-1][j], T2[i][j-1])

	jesli x[i] == y[j] && x[i] == `g`:
		T2[i][j] = max(T1[i-1][j-1] + 1, T2[i-1][j-1])

	jesli x[i] == y[j] && x[i] == `n`:
		T2[i][j] = max(T1[i-1][j-1] + 1, T2[i-1][j-1])

	jesli x[i] == y[j] && x[i] != `g` &&  x[i] != `n`:
		T2[i][j] = T7[i-1][j-1] + 1

\end{lstlisting}
Funkcje obliczT3(i,j) do obliczT6(i, j) są analogiczne tylko zamieniamy literę do sprawdzenia na koleją z ciągu `egzamin'. \newline
\begin{lstlisting}[
    basicstyle=\tiny, %or \small or \footnotesize etc.
    tabsize=4
]
obliczT7(i,j):
	jesli i == 0 || j == 0:
		T7[i][j] = 0;

	jesli x[i] != y[j]:					(1)
		T7[i][j] = max(T7[i-1][j], T7[i][j-1])

	jesli x[i] == y[j] && x[i] == `n`:			(2)
		T7[i][j] = max(T6[i-1][j-1] + 1, T7[i-1][j-1])
	jesli x[i] == y[j]  && x[i] != `n`:			(3)
		T7[i][j] = T7[i-1][j-1] + 1
		
\end{lstlisting}
\textbf{(1)} - znaki na miejsach i oraz j są różne, więc naturalnie bierzemy pod uwagę maksymalne wyniki dla innych indeksów obu ciągów. \newline
\textbf{(2)} - znaki na miejsach i oraz j są równe `n` czyli możemy wybrać maksymalny wynik z \textbf{I.} T6 czyli najdłuższego ciągu nie kończacego się na `egzami` (tak, żeby po dodaniu `n` otrzymać wciąż prawidłowe rozwiązanie) oraz \textbf{II.} T7 nie biorącego pod uwagę znaków na i oraz j. 
\textbf{(3)} -  znaki na miejsach i oraz j są równe i różne `n`, więc wybieramy maksymalne rozwiązanie dla ciągu bez znaków na indeksach i oraz j następnie inkrementujemy wynik końcowy.\newline \newline
Każdą z tych funckji wykonujemy pokolei według indeksów i oraz j. To znaczy: 
\begin{lstlisting}[
    basicstyle=\tiny, %or \small or \footnotesize etc.
    tabsize=4
]
lst-egzamin(n,m):
	dla kazdego i = 0 do n - 1:
		dla kazdego j = 0 do m -1:
		obliczT1(i,j):
		obliczT2(i,j):
		obliczT3(i,j):
		obliczT4(i,j):
		obliczT5(i,j):
		obliczT6(i,j):
		obliczT7(i,j): 

\end{lstlisting}
\newpage
\textbf{Wynik},a dokładnie długość najdłuższego wspólnego podciągu nie zawierającego słowa `egzamin` jest w ${T_7}$[n-1][m-1] gdzie odowiednio \newline n - liczba znaków słowa x \newline  m - liczba znaków słowa y. \newline
Do otrzymania konkretnego ciągu potrzebujemy przejść od tyłu (to jest od $T_7[n-1][m-1]$) i wybierać największe wartośći według kryteriów tak jak w funkcjach T1-T7. 
Czyli tak naprawdę jak w większości przypadków odtwarzania rozwiązania algorytmów dynamicznych, tylko tutaj mamy więcej niż jedną tablicę do przejrzenia wyników. \newline \newline
\textbf{Złożoność czasowa.} \newline
Każda funkcja wykonuje się w czasie $O(n*m)$ dla długości ciągów x i y odpowiednio n i m. Sprawdzanie warunków w fukcji oraz przypisanie wartości mamy w czasie stałym.  Mamy 7 funkcji, co też jest stałą wartością. Czyli ogólniej \textbf{$O(n^2)$}. \newline
\textbf{Złożoność pamięciowa.} \newline
Siedem tablic $O(n*m)$ czyli również \textbf{$O(n^2)$}.

\end{document}
