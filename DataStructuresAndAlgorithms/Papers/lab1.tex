\documentclass[a4paper,10pt]{article}
\usepackage{tabto}
\usepackage{listings}
\usepackage{tikz}
\usepackage[T1]{fontenc}
\usepackage[polish]{babel}
\usepackage[utf8]{inputenc}
\usepackage{lmodern}
\selectlanguage{polish}
\usepackage{amsmath}
\usepackage{amsthm}
\author{Karol Cidyło}
\title {Zadanie 6 z listy 1.}
\usetikzlibrary{matrix}
\begin{document}
Karol Cidyło \newline \newline \newline
Zadanie 6 z listy 1. \newline \newline
Pseudokod:  \newline
\begin{lstlisting}
deleteMaxNumbers(t[], n) {
    count = 0;
    j = n/2;
    i = 0;
    while ((j < n) and (i < n / 2)) {
        if ( 2 * t[i] <= t[j) {
            count++;
             i++;
        }
        j++;
    }

    return count;
}
\end{lstlisting}
Rozwiązanie działa w czasie O(n) sprawdzimy każdy element z tablicy maksymalnie raz. \newline \newline
\textbf{Lemat.} \newline 
Istnieje optymalne rozwiązanie, które wykreśla pary liczb z ciągu w taki sposób, że pierwszy element z pary jest zawsze w pierwszej połowie ciągu, a drugi w drugiej połowie. \newline
\textbf{Dowód.} \newline 
Weźmy optymalne rozwiązanie i nazwijmy je OPT. \newline
Weżmy teraz takie pary liczb z rozwiązania optymalnego, które nie spełniają tego lematu. \newline
Nie możemy mieć sytuacji, w której pierwszy element z pary jest w drugiej połowie, a drugi w pierwszej. Wynika to ze sposobu parowania oraz tego, że mamy niemalejący ciąg liczb dodatnich.\newline
Rozważmy więc sytuacje, w której OPT wybrał pary liczb, których oba elementy należą do tej samej połowy tablicy. \newline
Niech to będzie para ($a_i$, $a_j$) dla pierwszej połowy tablicy oraz para ($a_k$, $a_l$) dla drugiej połowy tablicy. \newline Ze sposobu parowania oraz faktu, że mamy niemalejący ciąg liczb wynika:  \newline $2*a_i$  $<=$ $a_j$  oraz  $2*a_k$  $<=$ $a_l$, wiemy , że skoro $a_j$ jest w pierwszej połowie ciągu, $a_k$ w drugiej to $a_j$ $<=$ $a_k$. Czyli wynika też: $2*a_i$  $<=$ $a_k$ i $2*a_j$  $<=$ $a_l$ możemy zatem przekształcić pary wybrane przez OPT do par ($a_i$ , $a_k$) oraz ($a_j$, $a_l$). Po przekształceniu wciąż mamy taką samą liczbę par wykreślonych przez oba algorytmy. \newline
Rozważmy kolejną sytuacje, w której w algorytmie OPT liczba par wybranych w pierwszej połowie jest różna od liczby par wybranych w drugiej połowie. \newline
Weźmy przypadek gdzie w pierwszej połowie jest więcej par, a dokładnie o jedną więcej, niech to będzie para ($a_i$, $a_j$),  w takim razie w drugiej połowie musi być jakiś niesparowany element, niech to będzie $a_k$, $2*a_i$  $<=$ $a_j$ oraz z faktu, że ciąg jest niemalejący $2*a_i$  $<=$ $a_j$ $<=$ $a_k$, więc również $2*a_i$  $<=$ $a_k$, czyli możemy to przekształcić do pary ($a_i$ , $a_k$). W przypadku gdzie takich par jest więcej powtarzamy powyższe kroki i zawsze będziemy w stanie dopasować element z pierwszej połowy z niesparowanym elementem z prawej połowy. W przypadku odwrotnej sytuacji, gdzie w drugiej połowie jest więcej par stosujemy analogiczne rozwiązanie. W tych przypadkach będziemy w stanie przekształcić pary rozwiązania OPT nie tracąc przy tym żadnego wykreślenia. Algorytm, który wykreśla pary liczb z ciągu, gdzie pierwszy element jest w pierwszej połowie,a  drugi w drugiej wykreśla tyle samo par co algorytm OPT. 
\begin{flushright}
\qedsymbol 
\end{flushright}
Teraz wystarczy pokazać, że pary są dobierane w taki sposób, że pierwszy element z pary(z pierwszej połowy) jest dopasowany z najmniejszym możliwym elementem z drugiej połowy ciągu. \newline
Załóżmy, że  i < j < k < l  oraz j < n/2 i k >= n/2 \newline

Rozważamy takie wykreślone pary: \newline
 ($a_j$, $a_k$) oraz  ($a_i$, $a_l$)\newline
\begin{tikzpicture}[edge_style/.style={draw=black, ultra thick}]

    \begin{scope}[]
        \matrix[nodes={draw,circle}, column sep=1cm]{
            \node (A) {$a_i$}; &
            \node (B) {$a_j$}; &
            \node (C) {$a_k$}; &
            \node (D) {$a_l$}; \\
            };

	\draw [-] (B) to [out=45,in=135] (C);
	\draw [-] (A) to [out=45,in=135] (D);
    \end{scope}
\end{tikzpicture} \newline
W naszym algorytmie najpierw porównywalibyśmy ze sobą $a_i$ oraz $a_k$ zanim doszlibyśmy
do porównania $a_j$ oraz $a_k$ Więc też mielibyśmy wykreśloną parę ($a_i$ oraz $a_k$). Skoro 
$a_j$ mogliśmy wykreślić z $a_k$ to tak samo możemy wykreślić w parze z $a_l$\newline
\begin{tikzpicture}[edge_style/.style={draw=black, ultra thick}]

    \begin{scope}[]
        \matrix[nodes={draw,circle}, column sep=1cm]{
            \node (A) {$a_i$}; &
            \node (B) {$a_j$}; &
            \node (C) {$a_k$}; &
            \node (D) {$a_l$}; \\
            };
	
	\draw [-] (A) to [out=45,in=135] (C);
	\draw [-] (B) to [out=45,in=135] (D);

    \end{scope}
\end{tikzpicture}
\newline \newline
Korzystając z lematu oraz przekształceń takich jak powyżej możemy rozwiązanie algorytmu optymalnego przekształcić do takiego, które zwraca nasz algorytm dając ten sam wynik. Z tego wynika, że algorytm daje optymalne rozwiązanie.

\end{document}