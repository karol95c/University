\documentclass[a4paper,10pt]{article}
\usepackage{tabto}
\usepackage{listings}
\usepackage{tikz}
\usepackage{enumitem}
\usepackage[T1]{fontenc}
\usepackage[polish]{babel}
\usepackage[utf8]{inputenc}
\usepackage{lmodern}
\selectlanguage{polish}
\usepackage{amsmath}
\usepackage{amsthm}
\usepackage{MnSymbol}
\author{Karol Cidyło}
\title {Zadanie 2 z listy 3.}
\usetikzlibrary{matrix}
\begin{document}
Karol Cidyło \newline \newline \newline
Zadanie 2 z listy 3. \newline \newline
\textbf{Algorytm} \newline
\begin{lstlisting}
Procedure MaxM in(S:set)
if |S|= 1 then return {a1, a1}
else
    if |S|= 2 then return (max(a1, a2), min(a1, a2))
    else
        podziel S na dwa rownoliczne (z dokladnoscia do jednego elementu)
        podzbiory S1, S2
        (max1, min1) <- MaxM in(S1)
        (max2, min2) <- MaxM in(S2)
        return (max(max1, max2), min(min1, min2))
        

.
\end{lstlisting}
$T(n) = T(\lfloor \frac{n}{2} \rfloor) + T(\lceil \frac{n}{2}\rceil) + 2$ \newline \newline
\textbf{Lemat} \newline
Weźmy $ n = 2^i +k$, gdzie $k < 2^i$: \newline
$$T(n) =  \begin{cases} 
      \frac{3}{2} 2^i - 2 + 2k & \text{jeśli } k\leq 2^{i-1} \\
      2 \cdot 2^{i} - 2 + k & \text{jeśli } k > 2^{i-1} 
   \end{cases}
$$
\textbf{Dowód indukcyjny po i} \newline
Mając $ n = 2^i +k$, gdzie $k < 2^i$: \newline
\begin{enumerate}
	\item \textbf{Podstawa:} \newline
		Dla i = 1
		\begin{enumerate}[label=(\Alph*)]
		\item k = 0. $T(2^1 + 0) = T(2) = 3 \cdot 1 + 2 \cdot 0 - 2 = 1 $ 
		\item k = 1. $T(2^1 + 0) = T(2) = 3 \cdot 1 + 2 \cdot 1 - 2 =3 $
		\end{enumerate}
	\item \textbf{Założenie:} \newline
	Załóżmy, że dla i - 1 jest to spełnione, czyli dla $\frac{n}{2} = \frac{2^{i-1} + k} {2}$
	\item \textbf{Krok indukcyjny:} \newline
	\begin{enumerate}[label=(\Alph*)]
		\item $k \leq 2^{i-1}$ \newline 
			$\frac{n}{2} = \frac{2^{i-1} + k}{2}$ \newline
			$\lceil \frac{n}{2} \rceil = 2^{i-1} + \lceil \frac{k}{2} \rceil$ \newline
			$\lceil \frac{k}{2} \rceil \leq 2^{i-2}$, $\lfloor \frac{k}{2} \rfloor \leq 2^{i-2}$ \newline
			$T(n) = T(\lfloor \frac{n}{2} \rfloor) + T(\lceil \frac{n}{2}\rceil) + 2 = \newline 
			= \frac{3}{2} 2^{i-1} - 2 + 2\lfloor \frac{k}{2} \rfloor + \frac{3}{2} 2^{i-1} - 2 + 2\lceil \frac{k}{2} \rceil +  				2 = \frac{3}{2} 2^i - 2 + 2k$
		\item $k > 2^{i-1}$ \newline
			$ 2^{i-1} \less \lceil \frac{k}{2} \rceil \less 2^{i+1}$ , $ 2^{i-1} \less \lfloor \frac{k}{2} \rfloor \less 2^{i+1}$   \newline
			$T(n) = T(\lfloor \frac{n}{2} \rfloor) + T(\lceil \frac{n}{2}\rceil) + 2 = T(\lfloor \frac{2^i + k }{2} \rfloor) + T(\lceil \frac{2^i + k }{2}\rceil) + 2 = \newline
			= 4\cdot2^{i-2} + \lfloor \frac{k}{2} \rfloor - 2 +   4\cdot2^{i-2}+ \lceil \frac{k}{2}\rceil -2 + 2 =   2\cdot2\cdot(2^{i-2} + 2^{i-2})  + k - 2 = \newline
			=  2 \cdot 2^{i} - 2 + k$ \newline
		\end{enumerate}
	\end{enumerate}
\begin{flushright}
\qedsymbol 
\end{flushright}
		Oznaczmy przez G(n) funkcję, która wyznacza różnicę w liczbie porównań. W takim razie: \newline
		$$ G(n) = T(n) - \lceil \frac{3}{2} n - 2 \rceil $$ \newline
		Podstawiając $ n = 2^i +k$ otrzymujemy: \newline 
		$$ G(n) = T(n) - \lceil \frac{3}{2} 2^i + \frac{3}{2} k - 2 \rceil = T(n) - \frac{3}{2} 2^i + 2 - \lceil \frac{3}{2} k \rceil$$ \newline
		Wykorzystując $T(n)$ z lematu mamy:\newline
		
		$$G(n) = \begin{cases}
		    \lfloor \frac{k}{2} \rfloor & \text{jeśli } k\leq 2^{i-1} \\
		    2^{i - 1} - \lceil \frac{k}{2} \rceil & \text{jeśli } k > 2^{i-1} 
	    \end{cases}
		$$
\begin{itemize}
	\item Wartości dla, których algorytm wykonuje $\lceil \frac{3}{2} n - 2 \rceil $ porównań: \newline
	To są te wartości k, dla których $G(n) == 0$, czyli {0, 1, $2^i-1$}. \newline
	\item Maksymalna różnica ($G(n)$) jest wtedy, gdy $ k = 2^{i-1}$ czyli: $G(n) = 2^{i-2}$ \newline
	$n = 2^i + k = 2^i + 2^{i-1}$ \newline
	$G(n) = \frac{n}{6}$\newline
	\item
	\begin{lstlisting}
Procedure MaxM in(S:set)
if |S|= 1 then return {a1, a1}
else
    if |S|= 2 then return (max(a1, a2), min(a1, a2))
    if |S| mod 2 == 0 and |S|/2 mod == 1
        podziel S na dwa zbiory o liczbie elementow:
        |S|/2 + 1 oraz |S|/2 - 1
    else
        podziel S na dwa rownoliczne 
        (z dokladnoscia do jednego elementu)
        podzbiory S1, S2
    (max1, min1) <- MaxM in(S1)
    (max2, min2) <- MaxM in(S2)
    return (max(max1, max2), min(min1, min2))
\end{lstlisting}
\end{itemize}
		
\end{document}
